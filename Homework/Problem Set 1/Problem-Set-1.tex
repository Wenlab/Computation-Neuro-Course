\documentclass{article}
\usepackage{amsmath}
\usepackage{hyperref}
\title{Problem Set I}
\date{Due Thursday, Sep  28, 2023}

\newcommand{\abs}[1]{\lvert #1 \rvert}

\begin{document}
\maketitle

\section*{Note on the Homework}

Please read \href{https://github.com/Wenlab/Computation-Neuro-Course/wiki/%E4%BD%9C%E4%B8%9A%E6%8F%90%E4%BA%A4}{how to submit a homework} in our github page.

\section*{Optimal length of a dendritic arbor}
In the class, I have raised up a thought experiment about how to wire up a neural circuit of $N$ neurons with all-to-all potential connectivity. Each neuron will use wire with diameter $d = 0.3\ \mu m$. I have considered several models.
\begin{itemize}
\item {Point-to-point wires} 
\item {use axons that meander through the space to find the cell body of another neuron}
\item {use both axons and dendrites and they are treated symmetrically}
\item {dendrite would grow a spine with length $s$}
\end{itemize} 
Recall how I define potential connectivity between neurons, and recall how I make the scaling argument. Make a similar derivation for the last case, and check whether this argument makes sense. Show how the volume of the network depends on $N$, $d$ and $s$,
and how the total length of the dendrite $L$ from a single neuron depends on these parameters. 


\section*{Quantitative Analysis of Dendritic Morphology}

Attached you will find morphology data txt files (.swc) of one pyramidal dendrite, one Purkinjie dendrite, and one arbor from larval zebrafish. Use MATLAB or Python to write a simple program: 
\begin{itemize}
\item Load the data file. 
\item Plot and visualize the neuronal 3D arbor shape. 
\item write a function to compute the mean path length form a dendritic segment to the soma
\item Compute the spine reach zone area of the Purkinje dendrite, namely the gray area that can be reached by growing a spine on a dendrite (slide 44), assuming the spine length $s = 2 ~\mu m$. And compare the total spine reach zone area with the arbor area, which may be defined as the convex hull area that embed the 2D arbor. For 3D pyramidal dendrites, choose a random 2D projection of the 3D morphology. Perform the same analysis. 
\end{itemize}
In the swc file, each column has the following meaning (from left to right): segment index, segment type (cell body =1, dendrite = 3), x coordinate ($\mu$m), y coordinate ($\mu$m), z coordinate ($\mu$m), segment diameter ($\mu$m), father segment index (root index = -1). 


\section*{Optimal Orientation Map}

Consider neurons in primary visual cortex (V1), where each neuron has a preferred orientation tuning. In my slides, we use different colors to represent each neuron's orientation selectivity $\theta \in [-\pi/2, \pi/2]$, namely the activity of a neuron in response to a bar (or edge) is maximized at a given orientation. We ask for a given a connectivity rule what is the optimal spatial layout of the neurons that minimize the total wires. Let us consider a simplified case where neurons are uniformly distributed on a 2-dimensional sheet (or located on a 2-D lattice). 
\begin{itemize}
\item {Rule 1} Consider that each neuron will establish equal numbers of connections with neurons of all preferred orientations. Prove that the optimal arrangement is \emph{Salt and Pepper}, namely neurons with different orientation selectivity are intermingled. In other words, in this layout, neurons of each preferred orientation are equally represented at every location. 
\item{Rule 2} Consider a specific connectivity rule:
\[
c(\Delta\theta) = c(0) 
\begin{cases} 
\sqrt{1- \Delta^2/\theta_{max}^2} & \abs{\Delta \theta} < \theta_{max} \\
0 & \abs{\Delta \theta} > \theta_{max} \\
\end{cases}
\]
where $\Delta\theta$ is the orientation selectivity difference.  Show that in this case, the optimal arrangement is \emph{Icecube} such that the preferred orientation smoothly varies spatially, i.e., $\theta(\vec{r}) = k\vec{r}$, with an appropriate periodicity. 
\end{itemize}
\emph{Hint}: Think this problem from a geometric viewpoint. No complicated calculation is involved.




\end{document}