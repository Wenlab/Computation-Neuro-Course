\documentclass{article}
\usepackage{amsmath, amssymb}
\usepackage{hyperref}
\usepackage{graphicx}
\title{Problem Set 2}
\date{Due Wednesday, Nov 10, 2021}

\begin{document}
\maketitle


\section*{Hodgkin and Huxley Model}

The Hodgkin-Huxley model for generation of an action potential is constructed by a summation of leaky current, a delayed-rectified $\textrm{K}^{+}$ current, and a transient $\textrm{Na}^{+}$ current:
\begin{eqnarray}
\begin{aligned}
C_{m}\frac{dV}{dt}=-\bar{g}_{\textrm{K}}n^{4}(V-E_{\textrm{K}})-\bar{g}_{\textrm{Na}}m^{3}h(V-E_{\textrm{Na}})-\bar{g}_L(V-E_L)+I_{e}.  \\  
\frac{dn}{dt}=\alpha_n (1-n) - \beta_n n. \\
\frac{dm}{dt}=\alpha_m(1-m) - \beta_m m. \\
\frac{dh}{dt}=\alpha_h(1-h) - \beta_h h. \\
\end{aligned}
\end{eqnarray}
\\
\\
(a) Please simulate the dynamic equations and check whether it could generate action potentials. Below I will provide detailed parameter values used in Hodgkin and Huxley model. 
\[\alpha_n=\frac{0.01(V+55)}{1-\exp(-0.1(V+55))}, \ \beta_n=0.125\exp(-0.0125(V+65)),\]
\[\alpha_m=\frac{0.1(V+40)}{1-\exp(-0.1(V+40))},\ \beta_m=4\exp(-0.0556(V+65)),\]
\[\alpha_h=0.07\exp(-0.05(V+65)), \ \beta_h=\frac{1}{1+\exp(-0.1(V+35))}. \]
These rates have dimensions $\textrm{ms}^{-1}$. The maximal conductances and reversal potentials used in the model are $\bar{g}_{\textrm{K}} = 0.36\  \textrm{mS/mm}^2$, $\bar{g}_{\textrm{Na}} = 1.2\  \textrm{mS/mm}^2$,   $\bar{g}_{\textrm{L}} = 0.003\  \textrm{mS/mm}^2$, $E_\textrm{L} = -54.387\  \textrm{mV}$, $E_\textrm{K} = -77\  \textrm{mV}$, $E_\textrm{Na} = 50\  \textrm{mV}$, $C_m = 10\ \textrm{nF/mm}^2$.
\\
\\
(b) Show that there is a threshold current above which the system generates periodic pulses. Explore the frequency of the pulses as a function of current, just like what you did in the integrate-and-fire model.

\section*{FitzHugh-Nagumo Model}

The FitzHugh-Nagumo model 

\begin{eqnarray}
\dot{V} = V(a-V)(V-1) - w +I,\\
\dot{w} = bV -cw,
\end{eqnarray}
imitates H-H models or the $I_{Na,p} + I_K$-model by having cubic (N-shaped) nullclines, where parameter $a$ describes the shape of the cubic parabola $V(a-V)(V-1)$, and $b>0$, $c\ge 0$ describe the kinetics of the recovery variable $w$. $I$ is the external current.

\begin{enumerate}

\item[(a)] Determine nullclines of the model and draw the two-dimensional phase portrait (vector field) of the model using MATLAB. 
\item[(b)] Please analyze the stability of the fixed point $(0,0)$, and how they depends on the above parameters. plot the phase digram just like what we did in the class. What happens when the fixed point is not stable? 
\item[(c)] When would the FitzHugh-Nagumo model have two stable fixed points?

\end{enumerate}

\section*{limit cycle behavior with two neurons}

Let us consider a nonlinear network with limit cycle attractor with two neurons:

\begin{eqnarray}
\begin{aligned}
\tau \frac{dx}{dt} = -x + \tanh(Jx - Ky)\\
\tau \frac{dy}{dt} = -y + Gx\\
\end{aligned}
\end{eqnarray}
where $J$, $K$, and $G >0$  are constants and $\tau$ is the time constant of the neurons. Note that x is an excitatory neuron and y is an inhibitory neuron. It is thus, a very generic model consisting of coupled activator and suppressor units. The system shows a variety of behaviors, including a stable limit cycle. 
\begin{enumerate}
\item[a.]	Find the fixed points of the system.  Find the set of parameters for which each fixed point exists.
\item[b.]	Consider the fixed point $(x,y) = (0,0)$.  Setting $\tau = 1$, sketch the phase diagram of the fixed point, plotting $J$ against $KG$.  In the $J-KG$ plane, draw and label the following four regions:
\begin{enumerate}
\item[i.]	The fixed point is stable and the system converges monotonically to the stable state under small perturbation.
\item[ii.]	The fixed point is stable and the system spirals back to the stable state under small perturbation.
\item[iii.]	The fixed point is unstable and the system diverges monotonically away under small perturbation.
\item[iv.]	The fixed point is unstable and the system spirals away under small perturbation.
\end{enumerate}
\item[c.]	Consider the case in which $(x,y) = (0,0)$ is unstable and the other fixed points do not exist. Find a closed surface in the x-y plane along which all the flow points inwards.  Qualitatively describe how the system evolves at long times in this case, given initial conditions inside the surface you found.
\item[d.]	Draw a new phase diagram in the $J-KG$ plane for the entire system, classifying when the system has 1 stable fixed point, when it has 2 stable fixed points, and when it produces a limit cycle.
\item[e.]	Numerically integrate this system of equations for $J$, $K$, and $G$ parameters representative of each of the four cases in (b), one of which will also be the case discussed in (c). Show how the system evolves in each case starting from an initial condition near $(0,0)$. Make sure you integrate long enough to see a steady state behavior emerge.  (You should hand in graphs of the $x-y$ plane.  For your own edification, plot $x(t)$ vs. $t$ and $y(t)$ vs. $t$ to make sure the system has reached a steady state.  If you use Matlab, ode45 is a useful integration command).

\end{enumerate}

\section*{Poisson Spikes}

For a constant rate Poisson process, every specific (up to a finite resolution) sequence of $N$ spikes occurring over a given time interval is equally likely. This seems paradoxical because we certainly do not expect to see all $N$ spikes appearing within the first 1 percent of the time interval. Resolve this paradox.


\section*{Analyze Motion Detection Neuron in Fly}

The MATLAB file H1.mat contains data collected and provided by Rob de Ruyter van Steveninck from a fly H1 neuron responding to an approximate white-noise visual motion stimulus. Data were collected for 20 minutes at a sampling rate of 500 Hz. In the file, $rho$ is a vector that gives the sequence of spiking events or nonevents at the sampled times (every 2 ms). When an element of rho is one, this indicates the presence of a spike at the corresponding time, whereas a zero value indicates no spike. The variable $stim$ gives the sequence of stimulus values at the sampled times. 
\begin{enumerate}
\item[(a)] Calculate and plot the spike-triggered average from these data over the range from 0 to 300 ms (150 time steps).

\item[(b)] Use the spike-triggered average to construct a linear kernel, as we discussed in the class, to provide a model of the response of the H1 neuron. Choose $r_0$ so that the average firing rate predicted by the model in response to the stimulus used
for the data matches the actual average firing rate. 

\item[(c)] Use a Poisson generator with the computed rate to generate a synthetic spike train from this linear estimate of the firing rate in response to the stimulus. Plot examples of the actual and synthetic spike trains. How
are they similar and how do they differ? 

\item[(d)]Plot the autocorrelation function of the actual and the synthetic spike trains over the range 0 to 100 ms. Why is there a dip at a lag of 2 ms in the autocorrelation of the actual spike train? Is there a dip for the synthetic train too? Plot the interspike interval histogram for both spike trains. Why is there a dip below 6 ms in the histogram for the actual spike train? What are the coefficents of variation (see lecture note) for the two spike trains and why might they differ? 

\end{enumerate}

\end{document}